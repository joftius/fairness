%%%%%%%%%%%%%%%%%%%%%%%%%%%%%%%%%%%%%%%%%%%%%%%%%%%%%%%%%%%%%%%%%%
%%%%%%%% ICML 2017 EXAMPLE LATEX SUBMISSION FILE %%%%%%%%%%%%%%%%%
%%%%%%%%%%%%%%%%%%%%%%%%%%%%%%%%%%%%%%%%%%%%%%%%%%%%%%%%%%%%%%%%%%

% Use the following line _only_ if you're still using LaTeX 2.09.
%\documentstyle[icml2017,epsf,natbib]{article}
% If you rely on Latex2e packages, like most moden people use this:
\documentclass{article}

% use Times
\usepackage{times}
% For figures
\usepackage{graphicx} % more modern
%\usepackage{epsfig} % less modern
\usepackage{subfigure} 

% For citations
\usepackage[sort&compress]{natbib}

% For algorithms
\usepackage{algorithm}
\usepackage{algorithmic}
\usepackage{amsthm,amssymb,amsopn,amsmath}

% As of 2011, we use the hyperref package to produce hyperlinks in the
% resulting PDF.  If this breaks your system, please commend out the
% following usepackage line and replace \usepackage{icml2017} with
% \usepackage[nohyperref]{icml2017} above.
\usepackage{hyperref}

% Packages hyperref and algorithmic misbehave sometimes.  We can fix
% this with the following command.
\newcommand{\theHalgorithm}{\arabic{algorithm}}

% Employ the following version of the ``usepackage'' statement for
% submitting the draft version of the paper for review.  This will set
% the note in the first column to ``Under review.  Do not distribute.''
\usepackage{icml2017} 

% Employ this version of the ``usepackage'' statement after the paper has
% been accepted, when creating the final version.  This will set the
% note in the first column to ``Proceedings of the...''
%\usepackage[accepted]{icml2017}
\newtheorem{assumption}{Assumption}
\newtheorem{define}{Definition}
\newtheorem{thm}{Theorem}
\newtheorem{lem}{Lemma}
\newtheorem{coro}{Corollary}

% The \icmltitle you define below is probably too long as a header.
% Therefore, a short form for the running title is supplied here:
\icmltitlerunning{Counterfactual Fairness}

\begin{document} 

\twocolumn[
\icmltitle{Counterfactual Fairness}

% It is OKAY to include author information, even for blind
% submissions: the style file will automatically remove it for you
% unless you've provided the [accepted] option to the icml2017
% package.

% list of affiliations. the first argument should be a (short)
% identifier you will use later to specify author affiliations
% Academic affiliations should list Department, University, City, Region, Country
% Industry affiliations should list Company, City, Region, Country

% you can specify symbols, otherwise they are numbered in order
% ideally, you should not use this facility. affiliations will be numbered
% in order of appearance and this is the preferred way.
\icmlsetsymbol{equal}{*}

\begin{icmlauthorlist}
\icmlauthor{Cieua Vvvvv}{goo}
\icmlauthor{Iaesut Saoeu}{ed}

\end{icmlauthorlist}

\icmlaffiliation{goo}{Googol ShallowMind, New London, Michigan, USA}
\icmlaffiliation{ed}{University of Edenborrow, Edenborrow, United Kingdom}

\icmlcorrespondingauthor{Cieua Vvvvv}{c.vvvvv@googol.com}

% You may provide any keywords that you 
% find helpful for describing your paper; these are used to populate 
% the "keywords" metadata in the PDF but will not be shown in the document
\icmlkeywords{Causality,Counter-Factual,Fairness}

\vskip 0.3in
]

% this must go after the closing bracket ] following \twocolumn[ ...

% This command actually creates the footnote in the first column
% listing the affiliations and the copyright notice.
% The command takes one argument, which is text to display at the start of the footnote.
% The \icmlEqualContribution command is standard text for equal contribution.
% Remove it (just {}) if you do not need this facility.

\printAffiliationsAndNotice{}  % leave blank if no need to mention equal contribution
%\printAffiliationsAndNotice{\icmlEqualContribution} % otherwise use the standard text.
%\footnotetext{hi}

\begin{abstract} 
% fairness is important
  People have begun to make use of machine learning techniques to
  automate decisions that, historically, have been unfairly biased
  against certain subgroups in the population (e.g., based on race,
  gender, sexual orientation). % This includes making loan decisions,
  % credit scoring, hiring employees, and crime monitoring, among many
  % others.
  Because the historical data is often biased, machine learning
  techniques that are used to make future predictions must account for
  this to avoid perpetuating discriminatory practices. There have been
  a number of recent works towards designing fair classifiers, however
  it is often unclear when to prefer one method over another. In this
  paper, we develop a framework for modeling fairness in any dataset
  using tools from counterfactual inference. We propose a definition
  called \emph{counterfactual fairness} that naturally captures the
  intuition that a decision is fair towards an individual, if it gives
  the same predictions in (a) in the observed world and (b) a world
  where the individual had always belonged to a different demographic
  group. We demonstrate our modeling framework on two real-world
  problems: 1. fair prediction of law school success and 2. fair
  modeling of an individual's latent criminality in stop-and-frisk
  policing data.
\end{abstract} 

\section{Introduction}
\label{introduction}
% ml is now everywhere
Machine learning is now used in fields as diverse as credit scoring (CITE), crime prediction (CITE), and loan assessment (CITE). As machine learning enters these new areas it is necessary for the modeler to think beyond the simple objective of maximizing prediction accuracy.

% in these new ml fields, we cannot discriminate
% discrimination can happen in multiple ways
% - direct discrimination
In particular, for many of these new applications, it is crucial to consider whether the predictions of a machine learning model are fair. For instance, imagine a bank wishes to train a machine learning model to predict whether or not an individual should be given a loan to buy a house. The bank wishes to use historical lending data on whether or not a loan was paid back, along with personal information on individuals. If the bank simply tries to learn a model that accurately predicts who to loan to solely based on whether the loan will be paid back, it may unjustly favor giving loans to applicants of a particular race. This could be due to many factors that are observed and unobserved in the features of a given dataset such as:
\begin{itemize}
\item (\emph{observed}) race could be a feature in the dataset, thus any algorithm using this feature for prediction is directly using race to discriminate.
\item (\emph{observed}) other features could be proxies for race, such as where an individual currently lives.
\item (\emph{unobserved}) there may exist historical biases that make it more difficult for certain races to secure employment, thereby making a direct comparison between two individuals of different races unfair and could actually harm prediction (as a high-earning individual in one race may have had to work much harder to obtain work than a similar individual in another race, thus it may be crucial to look beyond observed features and favor the hard-working individual).
\item (\emph{unobserved}) the historical bank data may favor giving loans to individuals of a certain race because of prejudiced lenders.
\end{itemize}
These are just a few possible sources of unfairness that an unaware classifier could exploit in order to make more accurate predictions.

% there's a lot of interest in this
There has been immense interest recently in designing machine learning algorithms that make fair predictions. (CITE A MILLION PAPERS). In large part each work focuses on formalizing fairness into a concrete definition that can be tested, and for which algorithms can be developed. 

By defining fairness concretely, one can design specific algorithms to satisfy such definitions. The hope is that such algorithms can begin to address the calls of various governing bodies about the need for fairness in automated algorithms. For instance, in the United States the Obama Administration has issued two reports (CITE), the first warning individuals about ``the potential of encoding discrimination in automated decisions" and the second describing ``how technologies can deliberately or inadvertently perpetuate, exacerbate, or mask discrimination"\footnote{https://obamawhitehouse.archives.gov/blog/2016/05/04/big-risks-big-opportunities-intersection-big-data-and-civil-rights}. Thus there is significant interest defining fairness in order to address it.

% a lot of this work just proposes a new definition of fairness and checks it
% these definitions may or may not be appropriate for a given problem
In large part, the initial work on fairness in machine learning has focused on formalizing the above definitions and using them to solve a discrimination problem in a certain dataset. Unfortunately, for a practitioner, law-maker, judge, or anyone else who is interested in implementing algorithms that control for discrimination, it can be difficult to decide which definition of fairness to choose for the task at hand. Indeed, we demonstrate that depending on the relationship between a sensitive attribute and the data certain definitions of fairness can actually \emph{increase discrimination}.

% we propose a way to model data that allows a practitioner to assess what definitions of fairness are right for the problem at hand, and algorithms to ensure fairness
% OR
% we propose a way to interpret fairness...
% a) relationship between fairness and causality
% b) use pearl's models
% c) having an explicit model allows us to test fairness with the assumptions laid bare
% tension: Pearl already talks about discrimination, so we aren't really inventing new models. Are we even new in using these models to talk about fairness? Maybe... Pearl talks about variables that we might want to compute counterfactuals for in order to see if discrimination is happening.  
% Our proposal is:
% - situate a sensitive variable in a graph (not new).
% - Look at old definitions and see if anything bad could happen (new). 
% - Then define counterfactual fairness (new). 
% - Modeling helps us see where the weaknesses are in our assumptions and definitions (maybe not new)
% We don't want to see if every definition is counterfactually fair because then we're like everyone else, saying our definition is best
% 

In this work, we describe how techniques from causal inference can be used to formalize questions of fair prediction. Specifically, we develop a technique to leverage the causal models of Pearl \cite{pearl2009causal} to model the relationship between the sensitive attribute and data. Our contributions are as follows:
\begin{itemize}
    \item We make use of causal modelling to analyze how previous definitions of fairness fare under different circumstances.
    \item We design a novel definition called \emph{counterfactual fairness} that is naturally naturally suited to the causal modelling of fairness.
\end{itemize}
We demonstrate that by explicitly representing fairness within a causal model it becomes easy to critique different definitions of fairness as well the prediction methods that aim to accomplish these notions of fairness.











% RETHINK SPIN, ALWAYS RETHINK



\section{Background}
\label{background}
%!TEX root=ricardo_draft.tex
We begin by describing the problem of fair prediction and introduce three of the most popular definitions developed for this task.  We then give background on causal modeling which will act as our `tool-kit' for modeling and defining fairness.

\subsection{Fairness}
Let's imagine a scenario in which predictions must be fair. For instance, imagine a university in the United States (US) would like to know how successful an applicant is going to be after graduation, call this $Y$, given their current incoming features $X$ such as test scores, grade-point average (GPA). To predict success a modeler is given a dataset of $n$ applications with features $\{X^{(1)}, \ldots, X^{(n)} \}$ and measures of graduation success $\{Y^{(1)}, \ldots, Y^{(n)}\}$. However, universities in the US have historically had racial and gender biases in student admission \cite{kane1998racial,kidder2000portia}. Thus, in addition we are given demographic and gender information for each individual $\{A^{(1)}, \ldots, A^{(n)}\}$ that we would like to use to ensure our model is fair.

What does it mean for a model to be fair? To answer this there has been a wealth of recent work aimed at defining fairness (CITE PAPERS). A few popular definitions are (a) Fairness Through Unawareness (FTU), (b) Demographic Parity (DP), (c) Equal Opportunity (EO), and (d) Individual Fairness (IF). These are defined as follows:

\begin{define}[Fairness Through Unawareness (FTU)]
An algorithm is fair so long as the sensitive attribute $A$ is not explicitly used in the decision-making process. Formally, any mapping $\hat{Y}: X \rightarrow Y$ satisfies this definition.
\end{define}

\begin{define}[Demographic Parity (DP)]
An algorithm is fair if its predictions are independent of the sensitive attributes $A$ across the population. Formally, a prediction $\hat{Y}$ satisfies this definition if, 
\begin{align}
P(\hat{Y} | A = 0) = P(\hat{Y} | A = 1). \nonumber
\end{align}
\end{define}

\begin{define}[Equal Opportunity (EO)]
An algorithm is fair if it is equally accurate for every value of the sensitive attribute $A$. Formally, a prediction $\hat{Y}$ satisfies this if,
\begin{align}
P(\hat{Y}=1 | A=0,Y=1) = P(\hat{Y}=1 | A=1,Y=1). \nonumber
\end{align}
\end{define}

\begin{define}[Individual Fairness (IF)]
  An algorithm is fair if it give similar predictions to similar individuals. Formally, if individuals $i$ and $j$ are similar then
\begin{align}
  \hat{Y}(X^{(i)}, A^{(i)}) \approx \hat{Y}(X^{(j)}, A^{(j)}).\nonumber
\end{align}
\end{define}


\subsection{Causal Models and Counterfactuals}
\label{subsec:cmc}
We will follow the framework of \citet{pearl:00}, where a causal
model is a triple $(U, V, F)$ of sets such that
\begin{itemize}
\item $U$ is a set of {\bf background} variables\footnote{These are
  sometimes called {\bf exogenous variables}, but the fact that members of $U$
  might depend on each other is not relevant to what follows.}, which are generated by factors
outside of our potential control;
\item $V$ is a set of {\bf endogenous} variables, where each member is determined by
  other variables in $U \cup V$;
\item $F$ is a set of functions $\{f_1, \dots, f_n\}$, one for each $V_i \in V$, such
that $V_i = f_i(pa_i, U_{pa_i})$, $pa_i \subseteq V \backslash
\{V_i\}$ and $U_{pa_i} \subseteq U$. Such equations are also known as
{\bf structural equations} \citep{bol:89}.
\end{itemize}

The notation ``$pa_i$'' is motivated by the extra assumption that the
model factorizes according to a directed acyclic graph (DAG). That is,
define a directed graph ${\mathcal G}=(U \cup V, \mathcal E )$ where each node corresponds to an
element of $U \cup V$, and each directed edge $V_i \leftarrow X$ is added if
and only if $X \in pa_i \cup U_{pa_i}$. By definition, $\mathcal G$ is
acyclic.

The model is causal in the sense that, for a given probability model
$p(U)$ for the background variables, it gives the distribution of a
subset of $V$ given an {\bf intervention} in another complementary
subset of $V$.  The operational meaning of an intervention on $V_i$ at
value $v$ is the substitution of the equation
$V_i = f_i(pa_i, U_{pa_i})$ with the equation $V_i = v$. This captures
the idea of an agent, external to the system, modifying it. For
instance, this may happen in a randomized controlled trial which
overrides the value of $V_i$ with a treatment that sets it at $v$, a
value chosen at random and independently of any other causes of the
system. The do-calculus of \citet{pearl:00} provides a way to identify
features of interventional distributions %, where possible,
using only estimates of the joint distribution of $V$ and knowledge of
the causal DAG.

Compared to the independence constraints given by a DAG, the full
specification of $F$ requires much stronger assumptions but also leads
to much stronger claims. In particular, it allows for the
calculation of {\bf counterfactual} quantities. % Without going into a
% detailed coverage of the topic
In brief, consider the following counterfactual
statement, ``the value of $Y$ if $X$ had taken value $x$'', for two endogenous
variables $X$ and $Y$ in a causal model. By assumption, the state of
any endogenous variable is fully determined by
the background variables and structural equations. The counterfactual is
modeled as the solution for $Y$ for a given $U = u$ where the equation(s)
for $X$ is (are) replaced with $X = x$.  We denote it by $Y_{X \leftarrow x}(u)$
\cite{pearl:00}, and sometimes as $Y_x$ if the context of the notation is clear.

Counterfactual inference, as specified by a causal model $(U, V, F)$,
is the computation of probabilities
$P(Y_{X \leftarrow x}(U)\ |\ W = w)$, where $W$, $X$ and $Y$ are
subsets of $V$. Inference proceeds in three steps, as explained in
more detail in Chapter 4 of \citet{pearl:16}:
\begin{enumerate}
\item Abduction: for a given prior on $U$, compute the posterior
  distribution of $U$ given the evidence $W = w$;
\item Action: substitute the equations for $X$ with the interventional
  values $x$, resulting in the modified set of equations $F_x$;
\item Prediction: compute the implied distribution on the remaining
  elements of $V$ using $F_x$ and the posterior $P(U\ | W = w)$.
\end{enumerate}



%%% Local Variables:
%%% mode: latex
%%% TeX-master: "ricardo_draft"
%%% End:


\section{Counterfactual Fairness}
\label{sec:count_fair}
% !TEX root=ricardo_draft.tex
%%% TeX-master: "ricardo_draft"~\ref{figure.simple_models}

% \begin{center}
  % \begin{tabular}{c}
  %   Here, a graph $U_A \rightarrow A \rightarrow Y \leftarrow U_Y$
  %   (rearrange it spatially in any way you want (including this table).)\\ (a)\\ \\
  %   Here, a graph made of edges \\
  %   $A \leftarrow U_A$, $Employed \leftarrow \{A, Prejudiced, Qualifications\}$,
  %   $Y \leftarrow \{Employed, U_Y\}$. \\ (b)\\ \\
  %   Here, a twin network,
  %   $A \leftarrow U_A$, $Employed \leftarrow \{A, Prejudiced, Qualifications\}$,
  %   $Y \leftarrow \{Employed, U_Y\}$,\\
  %   $Employed_a \leftarrow \{a, Prejudiced, Qualifications\}$,
  %   $Y_a \leftarrow \{Employed_a, U_Y\}$,\\
  %   $Employed_{a'} \leftarrow \{a', Prejudiced, Qualifications\}$,
  %   $Y_{a'} \leftarrow \{Employed_{a'}, U_Y\}$,
  %   (``$a$'' and ``$a'$'' are two extra nodes too)\\.
  %   \\ (c)\\
  % \end{tabular}
  % \label{fig:ex1}
  % \caption{(a) The graph corresponding to a causal model with $A$ being the protected outcome
  %   and $Y$ some outcome of interest, with background variables assumed to be independent.
  %   (b) Expanding the model to include an intermediate variable indicating whether the individual
  %   is employed with two (latent) background variables $\textbf{Prejudiced}$ (whether the person in charge
  %   of offering the job is prejudiced) and $\textbf{Qualifications}$ (a measure of the qualifications of
  %   the individual). (c) A twin network representation of this system \citep{pearl:00}
  %   under two different counterfactual levels for $A$. The network is created by creating copies
  %   of nodes descending from $A$, which inherit parents from the factual world that have not
  %   been affected.}
% \end{center}
%\end{figure}


%\subsection{Definition}
%Formally, a variable $Y$ is said to be counterfactually fair with
%respect to a protected attribute $A$ if
Given a predictive problem with fairness considerations, where $A$, $X$ and $Y$
represent the protected attributes, remaining attributes, and output of interest respectively,
let us assume that we are given a causal model $(U, V, F)$, where $V \equiv A \cup X$.
We postulate the following criterion for predictors of $Y$.
\begin{define}[Counterfactual fairness]
Predictor $\hat Y$ is {\bf counterfactually fair}
if under any context $X = x$ and $A = a$,
  \label{eq:cf_definition}
\begin{align}
  P(\hat Y_{A \leftarrow a\ }(U) = y\ |\ X = x, A = a)  =%\nonumber\\ 
  P(\hat Y_{A \leftarrow a'}(U) = y\ |\ X = x, A = a), 
\end{align}
for all $y$ and for any value $a'$ attainable by $A$.
\end{define}
%Simply put,

This is closely related to
{\bf actual causes} \cite{halpern:16}, or token causality in the sense
that, to be fair, $A$ should not be a cause of $\hat Y$ in any
individual instance. In other words, changing $A$ while holding things
which are not causally dependent on $A$ constant
will not change the distribution of $\hat Y$.
% \footnote{Notice that we always assume counterfactuals to be
%  well-defined by the model. For instance, ``race'' can be taken as a
%  surrogate for ``perceived race.''}
We also emphasize that
counterfactual fairness is an individual-level definition. This is
substantially different from comparing different individuals that happen to
share the same ``treatment'' $A = a$ and coincide on the values of
$X$, as discussed in Section 4.3.1 of \citep{pearl:16} and the
Supplementary Material\footnote{We strongly suggest that reviewers
  look at it.}. Differences between $X_a$ and $X_{a'}$ must be caused
by variations on $A$ only. Notice also that this definition is
agnostic with respect to how good a predictor $\hat Y$ is, which we
discuss in Section \ref{sec:methods}.

\noindent {\bf Relation to individual fairness}. IF is agnostic with
respect to its notion of similarity metric, which is both a strength
(generality) and a weakness (no unified way of defining similarity).
Counterfactuals and similarities are related, as in the classical
notion of distances between ``worlds'' corresponding to different
counterfactuals \cite{lewis:73}. Defining $\hat Y$ as a
deterministic function of $W \subset A \cup X \cup U$, as in several
of our examples to follow, then IF can be defined by treating equally two
individuals with the same $W$ in a way that is also counterfactually fair.

\noindent {\bf Relation to \citet{pearl:16}}.  In
Example 4.4.4 of \cite{pearl:16}, the authors condition instead on
$X$, $A$, and the observed realization of $\hat Y$, and calculate the
probability of the counterfactual realization $\hat Y_{A \leftarrow
  a'}$ differing from the factual.
%\footnote{The result is an expression
%  called the ``the probability of sufficiency'' for $A$, capturing the
%  notion that switching $A$ to a different value would be sufficient
%  to change $\hat Y$ with some probability.}.
This example conflates the predictor $\hat Y$ with the outcome $Y$, of
which we remain agnostic in our definition but which is used in the
construction of $\hat Y$ as in Section \ref{sec:methods}. Our framing
makes the connection to machine learning more explicit.
% We also emphasize that counterfactual fairness is an individual-level
% definition. This is substantially different from the notion of ``causal independence''
% as discussed in Section 4.3.1 of \cite{pearl:16}. Causal independence
% % requires
% % \begin{align}
% %   &P(\hat Y = y\ |\ do(A = a), W = w) =\nonumber\\ 
% %   &P(\hat Y = y\ |\ do(A = a'), W = w),
% % \end{align}
% % which
% entails comparing different units that happen to share the
% same ``treatment'' and coincide on values of $W$, while
% counterfactual fairness concerns the variation possible within an
% individual depending on their value of $a$ and the descendents of
% $A$ in the causal graph.

\begin{figure}
  \begin{tabular}{p{0.5\columnwidth}|p{0.5\columnwidth}}
    \centerline{\includegraphics[width=0.5\columnwidth]{implications_fig.pdf}}&
    \centerline{\includegraphics[width=0.5\columnwidth]{simple_models_no_q3}}(d)
  \end{tabular}
  \caption{\label{fig:ex1} {\bf Left:} (a) The graph corresponding to
    a causal model with $A$ being the protected attribute and $Y$ some
    outcome of interest, with background variables assumed to be
    independent.  (b) Expanding the model to include an intermediate
    variable indicating whether the individual is employed with two
    (latent) background variables $\textbf{Prejudiced}$ (if the person
    offering the job is prejudiced) and $\textbf{Qualifications}$ (a
    measure of the individual's qualifications). (c) A twin network
    representation of this system \citep{pearl:00} under two different
    counterfactual levels for $A$. This is created by copying nodes
    descending from $A$, which inherit unaffected parents from the
    factual world. (d) Two causal models for different
    real-world fair prediction scenarios.\label{figure.simple_models}
    See Section \ref{sec:count_fair} for discussion.}
\end{figure}

\subsection{Implications}
%
One simple but important implication of the definition of counterfactual fairness is the following:
%
\begin{lem}
  \label{lem:nondescend}
  Let $\mathcal G$ be the causal graph of the given model $(U, V, F)$.
  Then $\hat Y$ will be counterfactual fair if it is a function
  of the non-descendants of $A$.
\end{lem}
%
\begin{proof}
 Let $W$ be any non-descendant of $A$ in $\mathcal G$. Then $W_{A
   \leftarrow a}(U)$ and $W_{A \leftarrow a'}(U)$ have the same
 distribution by the three inferential steps in Section
 \ref{subsec:cmc}.  Hence, the distribution of any function $\hat Y$ of the non-descendants of $A$
 is invariant with respect to the counterfactual values of $A$.
\end{proof}

This does not exclude using a descendant $W$ of $A$ as a possible input to
$\hat Y$. However, this will only be possible in the case where the
overall dependence of $\hat Y$ on $A$ disappears, which will not
happen in general. Hence, Lemma~\ref{lem:nondescend} provides the most
straightforward way to achieve counterfactual fairness.

\noindent{\bf Ancestral closure of protected attributes.} Suppose that
a parent of a member of $A$ is not in $A$.  Counterfactual fairness
allows for the use of it in the definition of $\hat Y$. If this seems
counterintuitive, then we argue that the fault should be at the
postulated set of protected attributes rather than with the definition
of counterfactual fairness, and that typically we should expect set
$A$ to be closed under ancestral relationships given by the causal
graph. For instance, if {\it Race} is a protected attribute, and {\it
  Mother's race} is a parent of {\it Race}, then it should also be in
$A$.

\noindent{\bf Dealing with historical biases.} The explicit difference
between $\hat Y$ and $Y$ allows us to tackle historical biases. For
instance, let $Y$ be an indicator of whether a client defaults on a
loan, while $\hat Y$ is the actual decision of giving the
loan. Consider the DAG $A \rightarrow Y$, shown in Figure
\ref{fig:ex1}(a) with the explicit inclusion of set $U$ of independent
background variables. $Y$ is the objectively ideal measure for
decision making, the binary indicator of the event that the individual defaults on
a loan. If $A$ is postulated to be a protected attribute, then the
predictor $\hat Y = Y = f_Y(A, U)$ is not counterfactually fair, with the arrow $A
\rightarrow Y$ being (for instance) the result of a world that
punishes individuals in a way that is out of their control. Figure
\ref{fig:ex1}(b) shows a finer-grained model, where the path is
mediated by a measure of whether the person is employed, which is
itself caused by two background factors: one representing whether the
person hiring is prejudiced, and the other the employee's
qualifications. In this world, $A$ is a cause of defaulting, even if
mediated by other variables\footnote{For example, if the function
  determining employment $f_E(A,P,Q) \equiv I_{(Q > 0, P = 0 \text{ or } A
    \neq a)}$ then an individual with sufficient qualifications and
  prejudiced potential employer may have a different counterfactual
  employment value for $A = a$ compared to $A = a'$, and a different
  chance of default. }. The counterfactual fairness principle however
forbids us from using $Y$: using the twin network of \citet{pearl:00},
we see in Figure \ref{fig:ex1}(c) that $Y_a$ and $Y_{a'}$ need not be
identically distributed given the background variables.
  % \footnote{We assume
  % that function $f_Y(A, U)$ is not pathological, that is, it will give
  % different outcomes for different values of $A$ other things being
  % equal. Moreover, we assume interventions in $A$ are well defined.
  % For instance, ``race'' here could be formulated as ``race
  % perception'', which can be due to, for instance, to
% racially-associated names in a C.V. or loan application.}

In contrast, any function of variables not descendants of $A$ can be
used a basis for fair decision making. This means that any variable
$\hat Y$ defined by $\hat Y = g(U)$ will be counterfactually fair for
any function $g(\cdot)$. Hence, given a causal model, the functional
defined by the function $g(\cdot)$ minimizing some predictive error
for $Y$ will satisfy the criterion, as proposed in Section
\ref{sec:algorithm}. We are essentially learning a projection of $Y$
into the space of fair decisions, removing historical biases as a
by-product.

% There are two issues to be clarified at this point. First,
% it sounds potentially counter-intuitive that our definition
% seemingly allows for the use of (background) variable $\textbf{Prejudiced}$
% directly. However, our point of view is that there is no harm: this is
% a feature of some other person who judged a job application of the
% individual concerned, not a feature of the individual. This variable
% might prove itself to be a useless predictor of $Y$ depending on the
% structural equations, resulting from the marginalization of $A$, but
% in principle it is not harmful as implied by the model.

% The second and more complicated issue concerns the fact that
% However, it appears to be the case that $\hat Y$ and $A$ will be
% ``causally dependent'' in general given $W$, meaning that $P(\hat Y =
% y\ |\ do(A = a), W = w) \neq P(\hat Y = y\ |\ do(A = a'), W = w)$.
% Our argument on why this is not a problem mirrors the discussion in
% Section 4.3.1 of \cite{pearl:16}, on the differences between
% counterfactual exchangeability and causal independence: the latter is
% a comparison among different units that happen to share the same
% treatment and coincide on the same outcomes for $W$. But in the
% postulated model $W$ responds to $A$ so, starting from a baseline
% individual with treatment $do(A = a)$ and measurements $w$, consider
% the generative model to generate a comparable individual: under $do(A
% = a')$, perform rejection sampling among individuals until we find
% someone who matches our target on the same $w$. This sampling
% mechanism gives a different posterior distribution for $U$ than the
% within-individual distribution\footnote{For example, if $w$ is a
%   common outcome under $do(A = a)$ and $P(U)$, but rare under $do(A =
%   a')$ and $P(U)$.} of the original individual, which is the one we care about in our
% definition. Hence, the inequality
% \begin{align}
%   P(\hat Y = y\ |\ do(A = a), W = w)
%   \neq P(\hat Y = y\ |\ do(A = a'), W = w)
% \end{align}
% should not be a matter of
% concern for a criterion defined for individual level differences.

%%%%%%% ICML STUFF
%Note that we can build counterfactually fair
%predictive models for some $\hat Y$ even if the 
%structural equations that generated $Y$ are unfair. The idea is that we
%are learning a projection of $Y$ into an alternate world where it
%would be fair, which we may think of as a
%``closest world'' defined by our class of models and the
%causal structure of the world\footnote{The notion of ``closest world''
%  is pervasive in the literature of counterfactual inference under
%  different meanings \citep{pearl:00, halpern:16}.  Here, the cost
%  function used to map fair variables to unfair outcomes also plays a
%  role, but this concerns a problem dependent utility function that
%  would be present anyway in the unfair prediction problem, and is
%  orthogonal to the causal assumptions.}.
%%%%%%%% END OF ICML STUFF

\subsection{Further Examples}
\label{sec:further_examples}

To give further intuition for counterfactual fairness, we will consider
two real-world fair prediction scenarios: \textbf{insurance pricing}
and \textbf{crime prediction}. Each of these correspond to one of the
two causal graphs in Figure~\ref{figure.simple_models}(d). The
Supplementary Material provides a more mathematical discussion of
these examples with more detailed insights.

\paragraph{Scenario 1: The Red Car.}
A car insurance company wishes to price insurance for car
owners by predicting their accident rate $Y$. They assume there is an
unobserved factor corresponding to aggressive driving $U$, that (a)
causes drivers to be more likely have an accident, and (b) causes
individuals to prefer red cars (the observed variable $X$). Moreover,
individuals belonging to a certain race $A$ are more likely to drive
red cars. However, these individuals are no more likely to be
aggressive or to get in accidents than any one else. We show this in
Figure~\ref{figure.simple_models}(d) (\emph{Left}). Thus, using the
red car feature $X$ to predict accident rate $Y$ would seem to be an
unfair prediction because it may charge individuals of a certain race
more than others, even though no race is more likely to have an
accident. Counterfactual fairness agrees with this notion, as $X$
is a descendent of $A$ but $U$ is not. Interestingly, we can show
(Supplementary Material) that in a linear model, regression $Y$ on $A$
and $X$ is equivalent to regressing on $U$, so off-the-shelf
regression here is counterfactually fair. Regressing $Y$ on $X$ alone
obeys the FTA criterion but is not counterfactually fair, so
{\em omitting $A$ (FTU) may introduce unfairness into
  an otherwise fair world.}
%
\paragraph{Scenario 2: High Crime Regions.}
%%%%% ICML STUFF
%A local police precinct wants to know $Y$, whether a given house is to
%be broken into in any given day. The probability of $Y = 1$ depends on many
%unobserved factors ($U$) but also upon the neighborhood the house lies
%in ($X$). However, different ethnic groups are more likely to live in
%particular neighborhoods, and so neighborhood and break-in rates are
%often correlated with the race $A$ of the house occupier. This can be
%seen in Figure~\ref{figure.simple_models}(d) (\emph{Center}). Unlike the
%previous case, a predictor $\hat Y$ trained using $X$ and $A$ is not
%counterfactually fair. The only change from Scenario 1 is that now $Y$
%depends on $X$ as follows: $Y \!=\! \gamma U + \theta X$. Now if we
%solve for $\lambda_X,\lambda_A$ it can be shown that $\hat Y(X,a)
%\!=\! (\gamma - \frac{\alpha^2 \theta v_A}{\beta v_U})U + \alpha
%\theta a$. As this predictor depends on the values of $A$, $\hat
%Y(X,a) \!\neq\! \hat Y(X,a')$ and thus $\hat Y(X,A)$ is not
%counterfactually fair.
%%%%% END ICML STUFF
A city government wants to estimate crime rates by neighborhood to
allocate policing resources. Its analyst constructed training data
by merging (1) a registry of residents containing their neighborhood $X$
and race $A$, with (2) police records of arrests, giving each resident a
binary label with $Y = 1$ indicating a criminal arrest record.
Due to historically segregated housing, the location $X$ 
depends on $A$.
Locations $X$ with more police resources have larger numbers of
arrests $Y$.
And finally, $U$ represents the totality of socioeconomic factors
and policing practices that both influence where an individual may
live and how likely they are to be arrested and charged.
This can all be seen in Figure~\ref{figure.simple_models}(d)
(\emph{Right}).

In this example,
higher observed arrest rates in some neighborhoods
are due to greater policing there, not because people of different races
are any more or less likely to break the law.
The label $Y = 0$ does not
mean someone has never committed a crime, but rather that they have not
been caught.
{\em If individuals in the training data have not already had equal
  opportunity, algorithms enforcing EO will not remedy such unfairness}.
In contrast, a counterfactually fair approach would model
differential enforcement rates using $U$ and base predictions
on this information rather than on $Y$ directly.

% Unlike Scenario 1, $Y$ now
% depends on $X$ directly. If all structural equations are linear, then
% $U$ is a linear function of $A$ and $X$, and so, indirectly, a
% counterfactually fair $\hat Y$ can be expressed as a linear function
% of $A$ and $X$. However, this is different from assuming that $\hat Y$ can be
% \emph{any} linear combination of $A$ and $X$. As a matter of fact, the solution for the
% unrestricted least-squares regression of $Y$ on $A$ and $X$ cannot be
% written as a function of $U$ only, as shown in the Supplementary Material.
%Intuitively this is because the predictor $\hat Y(X,A)$ will be a
% function of not only $U$, but also of the part of $X$ that directly
% causes $Y$. This means that the change in $X$ caused by the
% counterfactual change in $A$ from $a$ to $a'$ will cause a change in
% $\hat Y(X,A)$ so that $\hat Y(X,a) \!\neq\! \hat Y(X,a')$.  In such
% a scenario, although the likelihood of a particular person having
% their house broken into does not depend upon race directly, it does
% vary which a persons race and so is not counterfactually fair.

In general, we need a
multistage procedure in which we first derive latent variables $U$, and
then based on them we minimize some loss with respect to $Y$. This is the
core of the algorithm discussed next.

%%%%%%%% ICML STUFF
%\paragraph{Scenario 3: University Success.}
%A university wants to know if students will be successful
%post-graduation $Y$. They have information such as: grade point
%average (GPA), advanced placement (AP) exams results, and other
%academic features $X$. The university believes however, that an
%individual's gender $A$ may influence these features and their
%post-graduation success $Y$ due to social discrimination. They also
%believe that independently, an individual's latent talent $U$ causes
%$X$ and $Y$. We show this in Figure~\ref{figure.simple_models}(d)
%(\emph{Right}). We can again ask, is the predictor $\hat Y(X,A)$
%counterfactually fair? In this case, the different between this and
%Scenario 1 is that $Y$ is a function of $U$ and $A$ as follows: $Y
%\!=\! \gamma U + \eta A$. We can again solve for $\lambda_X,\lambda_A$
%and show that $\hat Y(X,a) \!=\! (\gamma - \frac{\alpha \eta
%  v_A}{\beta v_U})U + \eta a$. Again $\hat Y(X,A)$ is a function of
%$A$ so it cannot be counterfactually fair.
%%%%%%%% END ICML STUFF

% TODO. Here the main
% definition is introduced, and how it relates to ``path deletion'',
% including the core example of $A \rightarrow X \rightarrow Y$, with
% two latent variables $U_x \rightarrow X$ and $U_y \rightarrow Y$,
% arguing that one might judge that the path from $A$ to $Y$ via is due
% to an unfair mechanism and that we need a notion of ``closest world''.


% When describing counterfactual fairness, it is important to
% distinguish between a counterfactually fair {\em world} in which the
% variable $Y$ we are interested in predicting is inherently
% counterfactually fair with respect to the protected attributes $A$,
% and a counterfactually fair {\em predictor} $\hat Y$, guaranteed to
% be counterfactually fair, regardless of the behavior of $Y$.

% Figure~\ref{figure.simple_models} shows possible worlds. {\em Left:} A
% counterfactually fair world in which the state of $Y$ has no
% dependencies on $A$. {\em Center and Right:} Potentially unfair worlds
% in which the state of $A$ can influence the state of $Y$ either
% indirectly as in {\em center} or directly as in {\em Right}.

%In Scenarios 2 and 3 in general $\hat Y(X,A)$ will also not satisfy demographic parity as $Y$ will be affected by $A$. In fact, we demonstrate in the next section that demographic parity is strictly more restrictive than counterfactual fairness.

%  The world on the right shows $Y'$ a
% causally fair variable that doesn't depend on $A$ being directly
% corrupted by a bias relating to $A$, resulting in an observation
% $Y$. In this case, learning a causally fair approximation of $Y$ can
% recover the true variable $Y'$ (up to noise).


% \subsection{Examples}
% To get an intuition for what the different definitions of fairness imply, we will
% revisit the examples of figure \ref{figure.simple_models}.
% begin by describing a few possible real-world scenarios. For each of these we will describe what counterfactually-fair and counterfactually-unfair predictors look like.

% \paragraph{Scenario 1: The Red Car.}
% Imagine a car insurance company wants a quick, anonymous way to determine how to price insurance for different car owners by predicting their accident likelihood $Y$. They've noticed that there is a correlation between driving a red car $X$ and a higher rate of automobile accidents. Thus they would like to increase the insurance for all red car drivers. 

% Imagine what's really going on is shown in Figure~\ref{figure.simple_models} (\emph{Left}). The correlation between have a red car $X$ and accidents $Y$ is due to an `aggressiveness' factor $U$: aggressiveness causes individuals to be in accidents more often, and it also attracts them to red cars. Unfortunately, the red car feature $X$ is also affected by an individual's race $A$. 

% TODO. Here the examples and their motivation can be as follows:

% \begin{itemize}
% \item something analogous to the red car example: $A$ is not a cause of
%   $Y$ but might indirectly bias the result even without using $A$ as a predictor;
% \item something with selection bias, maybe a toy version of COMPAS;
% \item something where an unfair judgment (say, credit score) that can be potentially
%   considered as a target variable, and an
%   ``objective target'' (say, defaulting on a loan) are present, and
%   what the recommendation is
% \end{itemize}
%%% Local Variables:
%%% mode: latex
%%% TeX-master: "ricardo_draft"
%%% End:



\section{Methods and Assessment}
\label{sec:methods}
% !TEX root=ricardo_draft.tex

As discussed in the previous Section, we need to relate $\hat Y$ to
$Y$ if the predictor is to be useful, and that we will restrict to
$\hat Y$ to be a (parameterized) function of the non-descendants of
$A$ in the causal graph. An algorithm is introduced in Section
\ref{sec:algorithm}, followed by a discussion of the assumptions that
can be used to express counterfactuals.

\subsection{Algorithm}
\label{sec:algorithm}

Let $\hat Y \equiv g_\theta(U, X_{\nsucc A})$ be a predictor
parameterized by $\theta$, such as a logistic regression or a neural
network, and where $X_{\nsucc A} \subseteq X$ are non-descendants of
$A$. Given a loss function $l(\cdot, \cdot)$ such as squared loss of
log-likelihood, and training data $\mathcal D \equiv \{(A^{(i)}, X^{(i)}, Y^{(i)})\}$
for $i = 1, 2, \dots, n$, we define $L(\theta) \equiv \sum_{i =
  1}^n \mathbb E[l(y^{(i)}, g_\theta(U^{(i)}, x^{(i)}_{\nsucc
    A})\ |\ x^{(i)}, a^{(i)}] / n$ as the empirical loss to be
minimized with respect to $\theta$.  Each expectation is with respect
to random variable $U^{(i)} \sim P_{\mathcal M}(U\ |\ x^{(i)},
a^{(i)})$ where $P_{\mathcal M}(U\ |\ x, a)$ is the conditional
distribution of the background variables as given by a causal model
$\mathcal M$ that is available by assumption. If this expectation
cannot be calculated analytically, Markov chain Monte Carlo (MCMC) can
be used to approximate it, resulting in the following algorithm.
  
\begin{algorithmic}[1]
\Procedure{FairLearning}{$\mathcal D, \mathcal M$}\Comment{Learned parameters $\hat \theta$}  
  \State For each data point $i \in \mathcal D$, sample $m$ MCMC samples
  $U_1^{(i)}, \dots, U_m^{(i)} \sim P_{\mathcal M}(U\ |\ x^{(i)},a^{(i)})$.
  \State Let $\mathcal D'$ be the augmented dataset where each point
  $(a^{(i)}, x^{(i)}, y^{(i)})$ in $\mathcal D$ is replaced with the corresponding $m$ points
  $\{(a^{(i)}, x^{(i)}, y^{(i)}, u_j^{(i)})\}$.
  \State $\hat \theta \leftarrow \mathrm{argmin}_\theta \sum_{i' \in \mathcal D'}
                                   l(y^{(i')}, g_\theta(U^{(i')}, x^{(i')}_{\nsucc A}))$.
\EndProcedure
\end{algorithmic}

At prediction time, we report $\tilde Y \equiv \mathbb E[\hat Y(U^\star,
  x^\star_{\nsucc A})\ |\ x^\star, a^\star]$ for a new data point $(a^\star,
x^\star)$.

\noindent{\bf Deconvolution perspective.} The algorithm can be
understood as a deconvolution approach that, given observables $A \cup
X$, extracts its latent sources and pipeline them into a predictive
model. We advocate that \emph{counterfactual assumptions must underlie
  all approaches that claim to extract the sources of variation of the
  data as ``fair'' latent components}. As an example,
\citet{louizos2015variational} start from the DAG $A \rightarrow X
\leftarrow U$ to extract $P(U\ |\ X, A)$. As $U$ and $A$ are not
independent given $X$ in this representation, a type of penalization
is enforced to create a posterior $P_{fair}(U\ | A, X)$ that is close
to the model posterior $P(U\ |\ A, X)$ while satisfying $P_{fair}(U\ |
A = a, X) \approx P_{fair}(U\ | A = a', X)$. But {\it this is neither
  necessary nor sufficient for counterfactual fairness}. The model for
$X$ given $A$ and $U$ must be justified by a causal mechanism, and
that being the case, $P(U\ |\ A, X)$ requires no postprocessing. As a
matter of fact, model $\mathcal M$ can be learned by penalizing
empirical dependence measures between $U$ and $A$ given $X$
(e.g. \citet{mooij:09}), but this concerns $\mathcal M$ and not $\hat Y$,
and is motivated by explicit assumptions about structural equations,
as described next.

\subsection{Designing the Input Causal Model}
\label{sec:limit-guide-model}

Model $\mathcal M$ must be provided to algorithm {\sc FairLearning}.
Causal models always require strong assumptions, but counterfactuals
in particular are typically presented in terms of structural equations
which are in general unfalsifiable. We point out that we do not need
to specify a fully deterministic model, and structural equations can
be relaxed as conditional distributions. In particular, the concept of
counterfactual fairness holds under three levels of assumptions of
increasing strength:

\noindent {\bf Level 1.}  Build $\hat Y$ using only the observable
non-descendants of $A$.  This only requires partial causal ordering
and no further causal assumptions, but in many problems there will be
few, if any, observables which are not descendants of demographic
factors.
  
\noindent {\bf Level 2.} We postulate background latent variables that
act as non-deterministic causes of observable variables, based on
explicit domain knowledge and learning algorithms\footnote{In some
  domains, it is actually common to build a model entirely around
  latent constructs with few or no observable parents nor connections
  among observed variables \citep{bol:89}.}. Information about $X$ is
passed to $\hat Y$ via $P(U\ |\ x, a)$.

\noindent {\bf Level 3.} We postulate a fully deterministic model with
latent variables. For instance, the distribution $P(V_i\ |\ V_1,
\dots, V_{i - 1})$ can be treated as an additive error model, $V_i
\!=\! f_i(V_1, \dots, V_{i - 1}) \!+\! e_i$ \citep{peters:14}. The
error term $e_i$ then becomes an input to $\hat Y$ as calculated from
the observed variables. This maximizes the information extracted by
the fair predictor $\hat Y$.

%%%%%%%%%%% ICML
%\begin{itemize}
%\item[Level 1] Given a causal DAG, build $\hat Y$
%  using as covariates only the observable variables  not
%  descendants of the protected attributes $A$. This
%  requires information about the DAG, but no
%  assumptions about structural equations or priors over background
%  variables. %  Here
%  % $\hat Y$ is not a function of $U$ but a function
%  % of the maximal subset of $X$ that does not contain descendants of $A$;
%\item[Level 2] Level 1 ignores much information, particularly if the protected
%  attributes are typical attributes such as race or sex,
%  which are parents of many other variables. To include information
%  from descendants of $A$, we postulate background latent variables
%  that act as causes of observable variables, based on explicit domain
%  knowledge and learning algorithms\footnote{In some domains, it is
%    actually common to build a model entirely around latent constructs
%    with few or no observable parents nor connections among observed
%    variables \citep{bol:89}.}. Information from $X$ will propagate to
%  the latent variables by conditioning.% As these
%  % variables are not descendants of $A$, they can be used to fairly predict
%  % $\hat Y$.  Conditioning on descendants of $A$ propagates information
%  % from $X$ to them. This dependency of each $V_i$ on its parents can be
%  % probabilistic.
%  % , instead of being given by a structural equation, as in the
%  % previous section;
% \item[Level 3] In Level 2, the model factorizes as a
%   general DAG, and each node follows a non-degenerate
%   distribution given observed and latent variables. % In the final
%   % set of assumptions,
%   In this level, we remove all randomness from the conditional
%   distributions obtaining a full decomposition $(U, V, F)$ of the
%   model. % Default assumptions %partially independent of the domain,
%   % might be invoked.
%   For instance, the distribution
%   $p(V_i\ |\ V_1, \dots, V_{i - 1})$ can be treated as an additive
%   error model, $V_i \!=\! f_i(V_1, \dots, V_{i - 1}) \!+\! e_i$
%   \citep{peters:14}. The error term $e_i$ then becomes an input
%   to $\hat Y$ after conditioning on the observed variables. This
%   maximizes the information extracted
%   by the fair predictor $\hat Y$.
%\end{itemize}
%%%%%%%%%% END ICML

%%% Local Variables:
%%% mode: latex
%%% TeX-master: "ricardo_draft"
%%% End:




\section{Experiments}
\label{sec:experiments}
% !TEX root=ricardo_draft.tex
% In this section we evaluate our framework for modeling fairness.
We illustrate our approach on a practical problem that requires
fairness, the \emph{prediction of success in law school}. A second
problem, \emph{separating actual and perceived criminality in police
  stops}, is described in the Supplementary Material. Following closely the
usual framework for assessing causal models in the machine learning
literature, the goal of this experiment is to quantify how our
algorithm behaves with finite sample sizes while assuming ground truth compatible
with a synthetic model.

\noindent {\bf Problem definition: Law school success}

% From 1991 to 1996
The Law School Admission Council
conducted a survey across 163 law
schools in the United States \cite{wightman1998lsac}. % The survey was
% designed to assess `the law school experience of minority students, as
% well as their ultimate entry into the profession'.
It contains information on 21,790 law students such as their entrance
exam scores (LSAT), their grade-point average (GPA) collected prior to
law school, and their first year average grade (FYA).
%, and following Law
%School i.e. whether students passed the final examination, the `bar
%exam' (P)).

Given this data, a school may wish to predict if an applicant will
have a high FYA.
% from information about their academic performance
% before law school.
The school would also like to make sure these
predictions are not biased by an individual's race and sex. However,
the LSAT, GPA, and FYA scores, may be biased due to social factors. % Our approach will use variables that are
% counterfactually fair for prediction.
We compare our framework with two unfair baselines: 1. \textbf{Full}:
the standard technique of using all features, including sensitive
features such as race and sex to make predictions;
2. \textbf{Unaware}: fairness through unawareness, where we do not use
race and sex as features. For comparison, we generate predictors $\hat
Y$ for all models using logistic regression.


\paragraph{Fair prediction.}
As described in Section~\ref{sec:limit-guide-model}, there are three
ways in which we can model a counterfactually fair predictor of
FYA. Level 1 uses any features which are not descendants of race and
sex for prediction. Level 2 models latent `fair' variables which are
parents of observed variables. These variables are independent of both
race and sex. Level 3 models the data using an additive error model,
and uses the independent error terms to make predictions. These models
make increasingly strong assumptions corresponding to increased
predictive power. We split the dataset 80/20 into a train/test set,
preserving label balance, to evaluate the models.

As we believe LSAT, GPA, and FYA are all biased by race and sex, we
cannot use any observed features to construct a counterfactually fair
predictor as described in Level 1. % Instead we would need to resort to a
% constant predictor% , such as the mean of FYA over the training set
% . % As this model is trivial we do not consider it. 

In Level 2, we postulate that a latent variable: a student's
\textbf{knowledge} (K), affects GPA, LSAT, and FYA scores. The causal
graph corresponding to this model is shown in
Figure~\ref{figure.law_school}, (\textbf{Level 2}). This is a
short-hand for the distributions:
% NIPS VERSION BELOW
% \[
% \begin{array}{cc}
%   \mbox{GPA} \sim {\cal N}(b_{G} + w_{G}^K K + w_{G}^R R + w_{G}^S S, \sigma_{G}),&  \hspace{0.2in}
%   \mbox{FYA} \sim {\cal N}(w_{F}^K K + w_{F}^R R + w_{F}^S S, 1),\\
%   \mbox{LSAT} \sim \textrm{Poisson}(\exp(b_{L} + w_{L}^K K + w_{L}^R R + w_{L}^S S)),& \hspace{0.2in}
%   \mbox{K} \sim {\cal N}(0,1)
% \end{array}
% \]
% ARXIV VERSION BELOW
\[
\begin{array}{ll}
  &\mbox{GPA} \sim {\cal N}(b_{G} + w_{G}^K K + w_{G}^R R + w_{G}^S S, \sigma_{G}),
  \;\;\mbox{FYA} \sim {\cal N}(w_{F}^K K + w_{F}^R R + w_{F}^S S, 1),\\
  &\mbox{LSAT} \sim \textrm{Poisson}(\exp(b_{L} + w_{L}^K K + w_{L}^R R + w_{L}^S S)), 
  \;\;\mbox{K} \sim {\cal N}(0,1)
\end{array}
\]
%\begin{align}
%\mbox{GPA} &\sim {\cal N}(b_{G} + w_{G}^K K + w_{G}^R R + w_{G}^S S, \sigma_{G}),
%\mbox{LSAT} &\sim \textrm{Poisson}(\exp(b_{L} + w_{L}^K K + w_{L}^R R + w_{L}^S S)) \nonumber \\
%\mbox{FYA} &\sim {\cal N}(w_{F}^K K + w_{F}^R R + w_{F}^S S, 1), 
%K &\sim {\cal N}(0,1) \nonumber
%\end{align}
% As FYA is already standardized to have mean $0$ and standard
% deviation $1$ we do not learn bias and standard deviation terms.
We perform inference on this model using an observed training set to
estimate the posterior distribution of $K$. We use the probabilistic
programming language Stan \cite{rstan} to learn $K$. We call the
predictor constructed using $K$, \textbf{Fair $K$}.


\begin{figure}[th]
  \begin{tabular}{p{0.5\columnwidth}p{0.5\columnwidth}}
    \centerline{\includegraphics[width=0.5\columnwidth]{law_school_model}}
    &
      \centerline{\includegraphics[width=0.5\columnwidth]{counterfactual}}
  \end{tabular}
  \caption{{\bf Left:} A causal model for the problem of predicting law school success fairly.\label{figure.law_school}
  {\bf Right:} Density plots of predicted $\mbox{FYA}_a$ and $\mbox{FYA}_{a'}$.\label{figure.counterfactual}
}
\end{figure}

\begin{table}
\centering
\caption{Prediction results using logistic regression. Note that we
  must sacrifice a small amount of accuracy to ensuring
  counterfactually fair prediction (Fair $K$, Fair Add), versus the
  models that use unfair features: GPA, LSAT, race, sex (Full,
  Unaware).} \label{table.pred_law}
\begin{tabular}{ccccc} 
\hline
 &  {\bf Full} & {\bf Unaware} & {\bf Fair $K$} & {\bf Fair Add} \\
\hline
RMSE & 0.873 & 0.894 & 0.929 & 0.918 \\
%\bf{Method} & %\multicolumn{2}{c}{\bf Full} & \multicolumn{2}{c}{\bf Unaware} & \multicolumn{2}{c}{\bf Fair L2} & \multicolumn{2}{c}{\bf Fair L3} \\
\hline
\end{tabular}
\end{table}

In Level 3, we model GPA, LSAT, and FYA as continuous variables with additive error terms independent of
race and sex (that may in turn be correlated with one-another). This model is shown in
Figure~\ref{figure.law_school}, (\textbf{Level 3}), and is expressed by: % the equations:
\begin{align}
\mbox{GPA} &= b_{G} + w_{G}^R R + w_{G}^S S + \epsilon_G, \;\; \epsilon_G \sim p(\epsilon_G) \nonumber \\
\mbox{LSAT} &= b_{L} + w_{L}^R R + w_{L}^S S + \epsilon_L, \;\; \epsilon_L \sim p(\epsilon_L) \nonumber \\
\mbox{FYA} &= b_{F} + w_{F}^R R + w_{F}^S S + \epsilon_F, \;\; \epsilon_F \sim p(\epsilon_F) \nonumber
\end{align}
We estimate the error terms $\epsilon_G,\epsilon_L$ by first fitting
two models that each use race and sex to individually predict GPA and
LSAT. We then compute the residuals of each model (e.g., $\epsilon_G
\!=\! \mbox{GPA} \!-\! \hat{Y}_{\scriptsize\mbox{GPA}}(R,S)$). We use
these residual estimates of $\epsilon_G,\epsilon_L$ to predict FYA. We
call this \emph{Fair Add}.


% impacts these features.

% We propose to model the law school data as shown in
% Figure~\ref{figure.law_school}. We suspect that variables race and sex
% affect student performance (e.g. GPA, LSAT, and FYA) due to factors
% such as cultural norms, which assume that individuals of a certain
% race or sex are `better suited' to be lawyers. Such beliefs could
% adversely impact students who do not fit these norms. Instead we would
% like to model the latent \emph{knowledge} (K) of a student, which also
% impacts these features. 
% We can then construct a predictor that
% predicts FYA fairly using knowledge. It is easy to show that such a predictor
% is counterfactually fair, whereas a predictor that uses features GPA and
% LSAT is not (in this case even including race and sex as
% features cannot correct this, as can be done in the linear case). The
% causal 
 %; %3. \textbf{Variational Fair Autoencoder (VFAE)} \cite{louizos2015variational}, a recent approach that works to learn a fair representation of the original data.
% compute counterfactuals for both race and sex

\paragraph{Accuracy.}
We compare the RMSE achieved by logistic regression for each of the
models on the test set in Table~\ref{table.pred_law}.  The
\textbf{Full} model achieves the lowest RMSE as it uses race and sex
to more accurately reconstruct FYA. Note that in this case, this model
is not fair even if the data was generated by one of the models shown
in Figure~\ref{figure.law_school} as it corresponds to Scenario 3. The
(also unfair) \textbf{Unaware} model still uses the unfair variables
GPA and LSAT, but because it does not use race and sex it cannot match
the RMSE of the \textbf{Full} model. As our models satisfy
counterfactual fairness, they trade off some accuracy. Our first model
\textbf{Fair $K$} uses weaker assumptions and thus the RMSE is
highest. Using the Level 3 assumptions, as in \textbf{Fair Add} we
produce a counterfactually fair model that trades
slightly stronger assumptions for lower RMSE.


\paragraph{Counterfactual fairness.}
We would like to empirically test whether the baseline methods are
counterfactually fair. To do so we will assume the true model of the
world is given by Figure~\ref{figure.law_school}, (\textbf{Level
  2}). We can fit the parameters of this model using the observed data
and evaluate counterfactual fairness by sampling from
it. Specifically, we will generate samples from the model given either
the observed race and sex, or \emph{counterfactual} race and sex
variables. We will fit models to both the original and counterfactual
sampled data and plot how the distribution of predicted FYA changes
for both baseline models. Figure~\ref{figure.counterfactual} shows
this, where each row corresponds to a baseline predictor and each
column corresponds to the counterfactual change. In each plot, the blue
distribution is density of predicted FYA for the original data and the
red distribution is this density for the counterfactual data. If a
model is counterfactually fair we would expect these distributions to
lie exactly on top of each other. Instead, we note that the
\textbf{Full} model exhibits counterfactual unfairness for all
counterfactuals except sex. We see a similar trend for the
\textbf{Unaware} model, although it is closer to being
counterfactually fair. To see why these models seem to be fair
w.r.t. to sex we can look at weights of the DAG which generates the
counterfactual data. Specifically the DAG weights from (male,female)
to GPA are ($0.93$,$1.06$) and from (male,female) to LSAT are
($1.1$,$1.1$). Thus, these models are fair w.r.t. to sex simply
because of a very weak causal link between sex and GPA/LSAT.

% here describe what we see

% maybe sample from model and check it out
%\paragraph{Model validity.}
 

% TODO rank top 10 students by ability or by other score in law_school.py which only considers observed features


% \begin{table}[t]
% \vspace{-2ex}
% \caption{}
% \vspace{-3ex}
% \label{table.pred_law}
% \begin{center}
% \resizebox{\columnwidth}{!}
% {
% \begin{sc}
% \footnotesize
% \begin{tabular}{c|c|c|c}
% \hline
% %\multicolumn{5}{c}{\textbf{Lower Bounds}}\\
% \hline
% & full & unaware  & fair l2 & fair l3 \\
% \hline
% RMSE & 0.873 & 0.894 & 0.929 & 0.918 \\ \hline
% \end{tabular}
% \end{sc}
% }
% \end{center}
% \vspace{-4ex}
% \end{table}

%{lr@{$\pm$}lr@{$\pm$}lr@{$\pm$}l}

%\subsection{Model criticism}
%%% Local Variables:
%%% mode: latex
%%% TeX-master: "ricardo_draft"
%%% End:


\section{Conclusion}
\label{sec:conclusion}
We have presented a new model of fairness we refer to as {\em
  counterfactual fairness}. This concept allows us to propose fair
algorithms that, rather than simply ignoring protected attributes, are
able to take into account the different opportunities and social biases
that may arise towards and individual of a particular race, gender, or
sexuality and compensate for these biases effectively. We
experimentally contrasted our approach with previous racially blind
approaches and show that our explicit causal models capture these
social biases and make clear the implicit trade-off between
prediction accuracy and fairness in an unfair world.

\bibliography{rbas,bibliography}
\bibliographystyle{icml2017}

\end{document} 


% This document was modified from the file originally made available by
% Pat Langley and Andrea Danyluk for ICML-2K. This version was
% created by Lise Getoor and Tobias Scheffer, it was slightly modified  
% from the 2010 version by Thorsten Joachims & Johannes Fuernkranz, 
% slightly modified from the 2009 version by Kiri Wagstaff and 
% Sam Roweis's 2008 version, which is slightly modified from 
% Prasad Tadepalli's 2007 version which is a lightly 
% changed version of the previous year's version by Andrew Moore, 
% which was in turn edited from those of Kristian Kersting and 
% Codrina Lauth. Alex Smola contributed to the algorithmic style files.  

%%% Local Variables:
%%% mode: latex
%%% TeX-master: "ricardo_draft"
%%% End:
