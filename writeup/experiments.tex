%!TEX root=ricardo_draft.tex
In this section we evaluate our framework for modeling fairness. We test our approach on two practical problems in which fairness must be enforced. For each problem we construct causal models, and make explicit how unfairness may affect observed and unobserved variables in the world. Given these models we are able to (a) derive counterfactually fair predictors and (b) predict latent variables such as a person's `criminality' (which may be useful for predicting crime) as well as their `perceived criminality' (which may be due to prejudices based on race and gender). Additionally, we can analyze how realistic our models are by comparing our observations with data generated from the model. We consider two real-world problems, the first is \emph{fair prediction of success in law school} and the second is \emph{separating actual and perceived criminality in stop-and-frisk data}.

\subsection{Law school success}
Between the years of 1991-1996 the Law School Admission Council counducted a survey following students across 163 law schools in the United States \cite{wightman1998lsac}. The survey was designed to assess `the law school experience of minority students, as well as their ultimate entry into the profession'. The survey contains students features before entering law school (e.g., their entrance exam (LSAT) scores and their undergraduate grade-point average (GPA)), during law school (e.g., their average grade of their first-year of law school (FYA)), and after (e.g., whether students passed the final examination, the `bar exam' (P)). 

Given this data a law school may wish to predict whether a law student will pass the final bar exam given information about their performance during and before law school. The school would also like to make sure these predictions are not biased by an individual's race and gender. However, we may be worried that our observed information about students, the LSAT, GPA, and FYA scores, are biased due to historical reasons by race and gender. Our approach will be to model these interactions as well as a latent variable that is counterfactually fair. We will then be able to use this latent variable as a feature for prediction.

We propose to model the law school data as shown in Figure~\ref{figure.law_school}. We suspect that variables race and sex affect student performance (e.g. GPA, LSAT, and FGA) due to factors such as cultural norms, which assume that individuals of a certain race or gender are `better suited' to be lawyers. Such beliefs could adversely impact students who do not fit these norms. Instead we would like to model the latent \emph{knowledge} (K) of a student, which also impacts these features as well as whether the student passes their final bar exam (P). We can then construct a predictor $\hat{Y}$ that predicts P fairly using knowledge and the testing location (LOC) which we suspect may have an affect on P. It is easy to show that $\hat{Y}$ is counterfactually fair, whereas a predictor that uses features GPA, LSAT, and FGA is not (in this case even including race and sex as features cannot correct this, as can be done in the linear case). The causal model is a short-hand for a system of equations:
\begin{align}
G &\sim {\cal N}(b_G + w_1 K + w_2 R + w_3 S, \sigma_G) \nonumber \\
L &\sim \textrm{Poisson}(\exp(b_L + w_4 K + w_5 R + w_6 S)) \nonumber \\
F &\sim {\cal N}(w_7 K + w_8 + w_9 R + w_{10} S, 1) \nonumber \\
P &\sim \textrm{Bernoulli}([1+\exp(-b_P + w_{11} K + w_{12} R + w_{13} S)]^{-1}) \nonumber \\
K &\sim {\cal N}(0,1) \nonumber
\end{align}

 

% TODO rank top 10 students by ability or by other score in law_school.py which only considers observed features

\begin{figure}[th]
\begin{center}
\vspace{-1ex}
\centerline{\includegraphics[width=\columnwidth]{law_school_model}}
\vspace{-2ex}
\caption{A causal model for the problem of predicting law school success fairly.\label{figure.law_school}}
\vspace{-2ex}
\end{center}
\end{figure}


\subsection{True vs. Perceived Criminality}

\begin{figure*}[th]
\begin{center}
\vspace{-1ex}
\centerline{\includegraphics[width=\textwidth]{law_school_model}}
\vspace{-2ex}
\caption{A causal model for the problem of predicting law school success fairly.\label{figure.law_school}}
\vspace{-2ex}
\end{center}
\end{figure*}

\subsection{Model criticism}