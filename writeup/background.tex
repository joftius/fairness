%!TEX root=ricardo_draft.tex
% We begin by describing the problem of fair prediction and introduce three of the most popular definitions developed for this task.  We then give a brief overview of causal modeling which will act as our `tool-kit' for modeling and defining fairness.

% \subsection{Fairness}
% Consider a scenario in which predictions must be fair. For instance, imagine a university in the United States (US) would like to know how successful an applicant is going to be after graduation, call this $Y$, given their current incoming features $X$ such as test scores, grade-point average (GPA). To predict success a modeler is given a dataset of $n$ applications with features $\{X^{(1)}, \ldots, X^{(n)} \}$ and measures of graduation success $\{Y^{(1)}, \ldots, Y^{(n)}\}$. However, historically  student admission \cite{kane1998racial,kidder2000portia} in universities in the US suffered from racial and gender biases. Thus, in addition we are given demographic and gender information for each individual $\{A^{(1)}, \ldots, A^{(n)}\}$ that we will use to ensure our model is fair.

% What does it mean for a model to be fair?  There has
% been a wealth of recent works aimed at answering this. A few popular
% definitions are (a) Fairness Through Unawareness
% (FTU)~\citep{dwork2012fairness,grgiccase}, (b) Demographic Parity (DP)
% \citep{kleinberg2016inherent}, (c) Equal Opportunity (EO)
% \citep{kleinberg2016inherent}, and (d) Individual Fairness (IF). These
% are defined as follows:

% \begin{define}[Fairness Through Unawareness (FTU)]
%   An algorithm is fair so long as the sensitive attribute $A$ is not
%   explicitly used in the decision-making process. Any mapping
%   $\hat{Y}: X \rightarrow Y$ that excludes $A$ (or other attributes
%   considered  unfair, see \citet{grgiccase}) satisfies this
%   definition.
% \end{define}

% \begin{define}[Demographic Parity (DP)]
% An algorithm is fair if its predictions are independent of the sensitive attributes $A$ across the population. A prediction $\hat{Y}$ satisfies this definition if, 
% \begin{align}
% P(\hat{Y} | A = 0) = P(\hat{Y} | A = 1). \nonumber
% \end{align}
% \end{define}

% \begin{define}[Equal Opportunity (EO)]
% An algorithm is fair if it is equally accurate for each value of the sensitive attribute $A$. A prediction $\hat{Y}$ satisfies this if,
% \begin{align}
% P(\hat{Y}=1 | A=0,Y=1) = P(\hat{Y}=1 | A=1,Y=1). \nonumber
% \end{align}
% \end{define}



%\subsection{Causal Models and Counterfactuals}
\label{subsec:cmc}
We  follow the framework of \citet{pearl:00}, and define a causal
model as a triple $(U, V, F)$ of sets such that
\begin{itemize}
\item $U$ is a set of {\bf background} variables\footnote{These are
  sometimes called {\bf exogenous variables}, but the fact that members of $U$
  might depend on each other is not relevant to what follows.}, which are generated by factors
outside of our potential control, and do not depend on any protected attributes $A$;
\item $V$ is a set of {\bf endogenous} variables, where each member is determined by
  other variables in $U \cup V$;
\item $F$ is a set of functions $\{f_1, \dots, f_n\}$, one for each $V_i \in V$, such
that $V_i = f_i(pa_i, U_{pa_i})$, $pa_i \subseteq V \backslash
\{V_i\}$ and $U_{pa_i} \subseteq U$. Such equations are also known as
{\bf structural equations} \citep{bol:89}.
\end{itemize}

The notation ``$pa_i$'' refers to the ``parents'' of $a_i$ and is motivated by the extra assumption that the
model factorizes according to a directed acyclic graph (DAG). That is, we can
define a directed graph ${\mathcal G}=(U \cup V, \mathcal E )$ where each node $X$ corresponds to an
element of $U \cup V$, and each directed edge from $X$ to $V_i$ is added if
and only if $X \in pa_i \cup U_{pa_i}$. By construction, $\mathcal G$ is
acyclic.

The model is causal in the sense that, given a  probability distribution
$p(U)$ over the background variables $U$, you can derive the distribution of a
subset of $X\subseteq V$ following an {\bf intervention} on the complementary
subset  $V\setminus X$.  The operational meaning of an intervention on the variable $V_i$ of
value $v$ is the substitution of the equation
$V_i = f_i(pa_i, U_{pa_i})$ with the equation $V_i = v$. This intervention captures
the idea of an agent, external to the system, modifying it by forcefully assigning value $v$ to $V_i$. For
instance, this occurs in a randomized controlled trial where the value of $V_i$ is overridden by a treatment that sets it to $v$, a
value chosen at random and thus independent of any other causes of the
system. The do-calculus of \citet{pearl:00} provides a way to identify
features of interventional distributions %, where possible,
using only estimates of the joint distribution of $V$ and knowledge of
the causal DAG.

In contrast with the independence constraints given by a DAG, the full
specification of $F$ requires much stronger assumptions but also leads
to much stronger claims. In particular, it allows for the
calculation of {\bf counterfactual} quantities. % Without going into a
% detailed coverage of the topic
In brief, consider the following counterfactual
statement, ``the value of $Y$ if $X$ had taken value $x$'', for two endogenous
variables $X$ and $Y$ in a causal model. By assumption, the state of
any endogenous variable is fully determined by
the background variables and structural equations. The counterfactual is
modeled as the solution for $Y$ for a given $U = u$ where the equations
for $X$ are replaced with $X = x$.  We denote it by $Y_{X \leftarrow x}(u)$
\cite{pearl:00}, and sometimes as $Y_x$ if the context of the notation is clear.

Counterfactual inference, as specified by a causal model $(U, V, F)$,
is the computation of probabilities
$P(Y_{X \leftarrow x}(U)\ |\ W = w)$, where $W$, $X$ and $Y$ are
subsets of $V$. Inference proceeds in three steps, as explained in
more detail in Chapter 4 of \citet{pearl:16}:
\begin{enumerate}
\item Abduction: for a given prior on $U$, compute the posterior
  distribution of $U$ given the evidence $W = w$;
\item Action: substitute the equations for $X$ with the interventional
  values $x$, resulting in the modified set of equations $F_x$;
\item Prediction: compute the implied distribution on the remaining
  elements of $V$ using $F_x$ and the posterior $P(U\ | W = w)$.
\end{enumerate}



%%% Local Variables:
%%% mode: latex
%%% TeX-master: "ricardo_draft"
%%% End:
