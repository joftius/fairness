\documentclass{article}

% if you need to pass options to natbib, use, e.g.:
\PassOptionsToPackage{numbers, sort&compress}{natbib}
% before loading nips_2017
%
% to avoid loading the natbib package, add option nonatbib:
% \usepackage[nonatbib]{nips_2017}

\usepackage{nips_2017}

% to compile a camera-ready version, add the [final] option, e.g.:
% \usepackage[final]{nips_2017}

\usepackage[utf8]{inputenc} % allow utf-8 input
\usepackage[T1]{fontenc}    % use 8-bit T1 fonts
\usepackage[pagebackref]{hyperref}       % hyperlinks
\usepackage{url}            % simple URL typesetting
\usepackage{booktabs}       % professional-quality tables
\usepackage{amsfonts}       % blackboard math symbols
\usepackage{nicefrac}       % compact symbols for 1/2, etc.
\usepackage{microtype}      % microtypography
\usepackage{algorithm}
\usepackage{algorithmic}
\usepackage{amsthm,amssymb,amsopn,amsmath}
\usepackage{graphicx} % more modern
%\usepackage{epsfig} % less modern
\usepackage{subfigure}
\usepackage{wrapfig}
%\usepackage[nonatbib]{nips_2016}
\newtheorem{assumption}{Assumption}
\newtheorem{define}{Definition}
\newtheorem{thm}{Theorem}
\newtheorem{lem}{Lemma}
\newtheorem{coro}{Corollary}
\title{Coutnerfactual Fairness}
%\newcommand\independent{\protect\mathpalette{\protect\independenT}{\perp}}
\def\independenT#1#2{\mathrel{\rlap{$#1#2$}\mkern2mu{#1#2}}}
\newcommand\cff{{\sc cff}}
% Code blocks%
\usepackage{listings}
\usepackage{color}

\newcommand\independent{\protect\mathpalette{\protect\independenT}{\perp}}

% Code blocks%
\usepackage{listings}
\usepackage{color}

\definecolor{dkgreen}{rgb}{0,0.6,0}
\definecolor{gray}{rgb}{0.5,0.5,0.5}
\definecolor{mauve}{rgb}{0.58,0,0.82}

\lstset{frame=tb,
  language=matlab,
  aboveskip=3mm,
  belowskip=3mm,
  showstringspaces=false,
  columns=flexible,
  basicstyle={\small\ttfamily},
  numbers=none,
  numberstyle=\tiny\color{gray},
  keywordstyle=\color{blue},
  commentstyle=\color{dkgreen},
  stringstyle=\color{mauve},
  breaklines=true,
  breakatwhitespace=true,
  tabsize=3
}


\begin{document}
\maketitle
\begin{abstract}
There has been immense recent interest in fairness in machine learning. Algorithms trained on data from the real world, which for historical reasons may produce unfair outcomes, will tend to perpetuate that unfairness through their predictions. For example, racial disparities in the US criminal justice system (cite) may be perpetuated through the predictions of recidivism risk assessment algorithms (cite). A number of recent papers have attempted to address this issue by defining various notions of fairness and proposing algorithms designed to give the best possible predictions while satisfying that definition of fairness. In this work, we leverage the language and methodology developed in the literature on causal inference to address fairness.
%For instance, imagine a bank wishes to train a machine learning model to predict whether or not an individual should be given a loan to buy a house. If the bank simply tries to learn a model that accurately predicts who to loan to solely based on whether the loan will be paid back. 
%Specifically, we would want any model that offers house loans to individuals to not be biased by an individual's race in granting such loans. Or we would like a
%with respect to race, gender, and any other individual attribute 
\end{abstract} 


\section{Introduction}
%!TEX root=ricardo_draft.tex
% ml is now everywhere
Machine learning has spread to fields as diverse as credit scoring
\cite{khandani2010consumer}, crime prediction
\cite{brennan2009evaluating}, and loan assessment
\cite{mahoney2007method}. Decisions in these areas may have ethical or
legal implications, so it is necessary for the modeler to think beyond
the objective of maximizing prediction accuracy and consider the
societal impact of their work.
% in these new ml fields, we cannot discriminate
% discrimination can happen in multiple ways
% - direct discrimination
For many of these applications, it is crucial to ask if the
predictions of a model are \emph{fair}.  Training data can contain
unfairness for reasons having to do with historical prejudices or
other factors outside an individual's control.
% For instance, imagine a bank
% wishes to predict if an individual should be given a loan to buy a
% house. The bank wishes to use historical repayment data, alongside
% individual data. If they simply learn a model that predicts whether
% the loan will be paid back, it may unjustly favor applicants of
% particular subgroups, due to past and present prejudices.
In 2016, the Obama administration released a
report\footnote{https://obamawhitehouse.archives.gov/blog/2016/05/04/big-risks-big-opportunities-intersection-big-data-and-civil-rights}
which urged data scientists to analyze ``how technologies can
deliberately or inadvertently perpetuate, exacerbate, or mask
discrimination."

There has been much recent interest in designing algorithms that make
fair predictions
\cite{hardt2016equality,dwork2012fairness,joseph2016rawlsian,kamishima2011fairness,zliobaite2015survey,zafar2016fairness,zafar2015learning,grgiccase,kleinberg:17,calders2010three,kamiran2012data,bolukbasi2016man,kamiran2009classifying,zemel2013learning,louizos2015variational}.
% Most
% of these works focus on formalizing fairness into a numeric
% definition and satisfying it with customized algorithms.
In large part, the literature has focused on formalizing fairness into
quantitative definitions and using them to solve a discrimination
problem in a certain dataset. Unfortunately, for a practitioner,
law-maker, judge, or anyone else who is interested in implementing
algorithms that control for discrimination, it can be difficult to
decide {\em which} definition of fairness to choose for the task at
hand. Indeed, we demonstrate that depending on the relationship
between a protected attribute and the data, certain definitions of
fairness can actually \emph{increase discrimination}.

% we propose a way to model data that allows a practitioner to assess what definitions of fairness are right for the problem at hand, and algorithms to ensure fairness
% OR
% we propose a way to interpret fairness...
% a) relationship between fairness and causality
% b) use pearl's models
% c) having an explicit model allows us to test fairness with the assumptions laid bare
% tension: Pearl already talks about discrimination, so we aren't really inventing new models. Are we even new in using these models to talk about fairness? Maybe... Pearl talks about variables that we might want to compute counterfactuals for in order to see if discrimination is happening.  
% Our proposal is:
% - situate a sensitive variable in a graph (not new).
% - Look at old definitions and see if anything bad could happen (new). 
% - Then define counterfactual fairness (new). 
% - Modeling helps us see where the weaknesses are in our assumptions and definitions (maybe not new)
% We don't want to see if every definition is counterfactually fair because then we're like everyone else, saying our definition is best
% 
% In this work,

In this paper, we introduce the first explicitly causal approach to
address fairness.  Specifically, we leverage the causal framework of
\citet{pearl2009causal} to model the relationship between protected
attributes and data. We describe how techniques from causal inference
can be effective tools for designing fair algorithms and argue, as in
\citet{dedeo2014wrong}, that it is essential to properly address
causality in fairness. In perhaps the most closely related prior work,
\citet{johnson2016impartial} make similar arguments but from a
non-causal perspective. An alternative use of causal modeling in
the context of fairness is introduced independently by \citep{kilbertus:17}.

In Section \ref{sec:background}, we provide a summary of basic
concepts in fairness and causal modeling. In Section
\ref{sec:count_fair}, we provide the formal definition of
\emph{counterfactual fairness}, which enforces that a distribution
over possible predictions for an individual should remain unchanged in
a world where an individual's protected attributes had been different
in a causal sense. In Section \ref{sec:methods}, we describe an
algorithm to implement this definition, while distinguishing it from
existing approaches.  In Section \ref{sec:experiments}, we illustrate
the algorithm with a case of fair assessment of law school success.

%%%%%%% ICML TEXT
%In this paper, we introduce the first explicitly formal causal model
%approach for casting problems of fair predictions with explicit
%counterfactual assumptions. We describe how techniques from causal
%inference can be effective tools for designing fair algorithms and
%argue, as in \citet{dedeo2014wrong}, that it is essential to properly
%address causality.  Specifically, we leverage the causal framework of
%\citet{pearl2009causal} to model the relationship between sensitive
%attributes and data. Our contributions are as follows:
%\begin{enumerate}
%    \item We model questions of fairness within a causal
%      framework. This allows us to directly model \emph{how}
%      unfairness affects the data at hand.
%    \item We introduce \emph{counterfactual fairness}, which enforces
%      that a distribution over possible predictions for an individual
%      should remain unchanged, in a world where an individual's
%      sensitive attribute had been different from birth.
%    \item We analyze how enforcing existing definitions of fairness
%      for different data may correspond or be in conflict with
%      counterfactual fairness. In particular, we show that depending
%      on the underlying state of the world some definitions of
%      fairness may be inappropriate.
%    \item We devise techniques for learning predictors that are
%      counterfactually fair and demonstrate their use in several
%      examples.
%\end{enumerate}
%%%%%%%% END ICML TEXT

%We demonstrate that by explicitly representing fairness within a causal model it becomes easy to critique different definitions of fairness as well the prediction methods that aim to accomplish these notions of fairness.




% RETHINK SPIN, ALWAYS RETHINK

%%% Local Variables:
%%% mode: latex
%%% TeX-master: "ricardo_draft"
%%% End:


\section{Fairness in machine learning}
%!TEX root=ricardo_draft.tex
% We begin by describing the problem of fair prediction and introduce three of the most popular definitions developed for this task.  We then give a brief overview of causal modeling which will act as our `tool-kit' for modeling and defining fairness.

% \subsection{Fairness}
% Consider a scenario in which predictions must be fair. For instance, imagine a university in the United States (US) would like to know how successful an applicant is going to be after graduation, call this $Y$, given their current incoming features $X$ such as test scores, grade-point average (GPA). To predict success a modeler is given a dataset of $n$ applications with features $\{X^{(1)}, \ldots, X^{(n)} \}$ and measures of graduation success $\{Y^{(1)}, \ldots, Y^{(n)}\}$. However, historically  student admission \cite{kane1998racial,kidder2000portia} in universities in the US suffered from racial and gender biases. Thus, in addition we are given demographic and gender information for each individual $\{A^{(1)}, \ldots, A^{(n)}\}$ that we will use to ensure our model is fair.

% What does it mean for a model to be fair?  There has
% been a wealth of recent works aimed at answering this. A few popular
% definitions are (a) Fairness Through Unawareness
% (FTU)~\citep{dwork2012fairness,grgiccase}, (b) Demographic Parity (DP)
% \citep{kleinberg2016inherent}, (c) Equal Opportunity (EO)
% \citep{kleinberg2016inherent}, and (d) Individual Fairness (IF). These
% are defined as follows:

% \begin{define}[Fairness Through Unawareness (FTU)]
%   An algorithm is fair so long as the sensitive attribute $A$ is not
%   explicitly used in the decision-making process. Any mapping
%   $\hat{Y}: X \rightarrow Y$ that excludes $A$ (or other attributes
%   considered  unfair, see \citet{grgiccase}) satisfies this
%   definition.
% \end{define}

% \begin{define}[Demographic Parity (DP)]
% An algorithm is fair if its predictions are independent of the sensitive attributes $A$ across the population. A prediction $\hat{Y}$ satisfies this definition if, 
% \begin{align}
% P(\hat{Y} | A = 0) = P(\hat{Y} | A = 1). \nonumber
% \end{align}
% \end{define}

% \begin{define}[Equal Opportunity (EO)]
% An algorithm is fair if it is equally accurate for each value of the sensitive attribute $A$. A prediction $\hat{Y}$ satisfies this if,
% \begin{align}
% P(\hat{Y}=1 | A=0,Y=1) = P(\hat{Y}=1 | A=1,Y=1). \nonumber
% \end{align}
% \end{define}



%\subsection{Causal Models and Counterfactuals}
\label{subsec:cmc}
We follow the framework of \citet{pearl:00}, and define a causal
model as a triple $(U, V, F)$ of sets such that
%
\begin{itemize}
\item $U$ is a set of {\bf background} variables\footnote{These are
  sometimes called {\bf exogenous variables}, but the fact that members of $U$
  might depend on each other is not relevant to what follows.}, which are generated by factors
outside of our potential control, and do not depend on any protected attributes $A$;
\item $V$ is a set of {\bf endogenous} variables, where each member is determined by
  other variables in $U \cup V$;
\item $F$ is a set of functions $\{f_1, \dots, f_n\}$, one for each $V_i \in V$, such
that $V_i = f_i(pa_i, U_{pa_i})$, $pa_i \subseteq V \backslash
\{V_i\}$ and $U_{pa_i} \subseteq U$. Such equations are also known as
{\bf structural equations} \citep{bol:89}.
\end{itemize}
%
The notation ``$pa_i$'' refers to the ``parents'' of $V_i$ and is motivated by the assumption that the
model factorizes according to a directed acyclic graph (DAG). That is, we can
define a directed graph ${\mathcal G}=(U \cup V, \mathcal E )$ where each node is an
element of $U \cup V$, and each edge from some $Z \subseteq U \cup V$ to $V_i$ indicates that $Z \in pa_i \cup U_{pa_i}$. By construction, $\mathcal G$ is
acyclic.

The model is causal in that, given a distribution $p(U)$
over the background variables $U$, you can derive the distribution of
a subset $Z \subseteq V$ following an {\bf intervention} on the
complementary subset $V\setminus Z$.  Here,
an {\bf intervention} on the variable $V_i$ of value $v$ refers to the substitution of
equation $V_i = f_i(pa_i, U_{pa_i})$ with the equation $V_i =
v$. This captures the idea of an agent, external to the
system, modifying it by forcefully assigning value $v$ to $V_i$. This occurs in a randomized controlled trials where the value
of $V_i$ is overridden by a treatment setting it to $v$, a value
chosen at random, and thus independent of any other causes.% of the
%system. % The do-calculus of \citet{pearl:00} provides a way to identify
% features of interventional distributions %, where possible,
% using only estimates of the joint distribution of $V$ and knowledge of
% the causal DAG.

In contrast with the independence constraints given by a DAG, the full
specification of $F$ requires much stronger assumptions but also leads
to much stronger claims. In particular, it allows for the
calculation of {\bf counterfactual} quantities. % Without going into a
% detailed coverage of the topic
In brief, consider the following counterfactual
statement, ``the value of $Y$ if $Z$ had taken value $z$'', for two endogenous
variables $Z$ and $Y$ in a causal model. By assumption, the state of
any endogenous variable is fully determined by
the background variables and structural equations. The counterfactual is
modeled as the solution for $Y$ for a given $U = u$ where the equations
for $Z$ are replaced with $Z \!=\! z$.  We denote it by $Y_{Z \leftarrow z}(u)$
\cite{pearl:00}, and sometimes as $Y_z$ if the context of the notation is clear.

Counterfactual inference, as specified by a causal model $(U, V, F)$ given evidence $W$,
is the computation of probabilities
$P(Y_{Z \leftarrow z}(U)\ |\ W \!=\! w)$, where $W$, $Z$ and $Y$ are
subsets of $V$. Inference proceeds in three steps, as explained in
more detail in Chapter 4 of \citet{pearl:16}:
\begin{enumerate}
\item Abduction: for a given prior on $U$, compute the posterior
  distribution of $U$ given the evidence $W = w$;
\item Action: substitute the equations for $Z$ with the interventional
  values $z$, resulting in the modified set of equations $F_z$;
\item Prediction: compute the implied distribution on the remaining
  elements of $V$ using $F_z$ and the posterior $P(U\ | W = w)$.
\end{enumerate}



%%% Local Variables:
%%% mode: latex
%%% TeX-master: "ricardo_draft"
%%% End:


\section{The counterfactual approach}
%!TEX root=main.tex
Our main idea is to model fairness using causal graphical
models\cite{pearl:09}. Specifically we model observed feature(s) $X$,
class label(s) $Y$, and sensitive attribute(s) $A$, as nodes in a
directed acyclic graph (DAG). A directed arrow from any node $C$ to
node $E$ indicates a causal relationship from cause $C$ to effect
$E$. Mathematically, we take the approach of structural equation
models as follows: if $C$ causes $E$ then $E$ is a function of
$C$. Additionally, each node has an associated noise variable that are
jointly independent


% and a noise variable associated with $C$, called $N_C$. (CITE) 


% a causal relation

%  where a directed arrow from


% that induce a joint distribution $P(X,Y,A)$. This distribution contains conditional independeces which can be expressed by a directed acyclic graph (DAG).









\begin{figure*}[th!]
\begin{center}
\vspace{-2ex}
\centerline{\includegraphics[width=\textwidth]{simple_models_no_q}}
\vspace{-2ex}
\caption{Three possible states of the world.\label{figure.simple_models}}
\vspace{-2ex}
\end{center}
\end{figure*}
%%% Local Variables:
%%% mode: latex
%%% TeX-master: "ricardo_draft"
%%% End:



\section{Unprejudicial Inference under Causal Structures}
\label{sec-1}

Much as we talk about variables being causally independent of one
another, it's possible to talk of predictions being counter factually
fair.

We say that a predictor of $Y$, $\hat Y(X,A)$ is causally
independent of $A$ if 
%
\[ \hat Y(X,A)=\hat Y(X,A)|\text{do}(A=a) \] 
%
In the general case, a classifier $\hat Y (X)$ that does not directly
depend on $A$ need not be causally independent of $A$ if $X$ depends on $A$.

We are interested in three related questions, illustrated by the
causal diagrams in figure 1. From left to right:

\begin{enumerate}
\item Given a $Y$ causally independent of $A$, can we learn a $\hat Y$,
causally independent of $A$ that accurately predicts $Y$?
\item Given a $Y$ causally \textbf{dependent} on $A$, can we learn a $\hat Y$,
causally independent of $A$ that predicts $Y$ as closely as possible?
\item Given a $Y'$, causally independent of $A$ but unobserved, and an
observed $Y$ which represents $Y$ corrupted by a function of $A$,
can we recover $Y'$?
\end{enumerate}

To simplify this problem, we first consider the linear case in where
variables are distributed Gaussianly and the dependencies are
additive, and related this to previous existing work on orthogonality
in fairness, before considering the more general case.

\subsection{Fair Learning in a Fair World}
\label{sec-1-1}
\subsection{Fair Learning in an Unfair World}
\label{sec-1-2}
\subsection{Recovering Fairness from Corrupted Data}
\label{sec-1-3}


\bibliography{bibliography}
\bibliographystyle{icml2017}

\end{document} 